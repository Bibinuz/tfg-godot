\documentclass[10pt,a4paper,twocolumn,twoside]{article}
\usepackage[utf8]{inputenc}
\usepackage[provide=*]{babel}
\usepackage{multicol}
\usepackage{graphicx}
\usepackage{fancyhdr}
\usepackage{times}
\usepackage{titlesec}
\usepackage{multirow}
\usepackage{lettrine}
\usepackage{microtype}
\usepackage[top=2cm, bottom=1.5cm, left=2cm, right=2cm]{geometry}
\usepackage[figurename=Fig.,tablename=TAULA]{caption}
%\captionsetup[table]{textfont=sc}

\author{\LARGE\sffamily Biel Alavedra Busquet}
\title{\Huge{\sffamily Engraginy: Simulació de Sistemes de Transmissió Mecànica}}

\newcommand\blfootnote[1]{%
  \begingroup
  \renewcommand\thefootnote{}\footnote{#1}%
  \addtocounter{footnote}{-1}%
  \endgroup
}

\titleformat{\section}
{\large\sffamily\scshape\bfseries}
{\textbf{\thesection}}{1em}{}

\begin{document}

\fancyhead[LO]{\scriptsize BIEL ALAVEDRA BUSQUET: ENGRAGINY}
\fancyhead[RO]{\thepage}
\fancyhead[LE]{\thepage}
\fancyhead[RE]{\scriptsize EE/UAB TFG INFORMÀTICA: ENGRAGINY (MECÀNICA I LOGÍSTICA)}

\fancyfoot[CO,CE]{}

\fancypagestyle{primerapagina}
{
   \fancyhf{}
   \fancyhead[L]{\scriptsize TFG EN ENGINYERIA INFORMÀTICA, ESCOLA D'ENGINYERIA (EE), UNIVERSITAT AUTÒNOMA DE BARCELONA (UAB)}
   \fancyfoot[C]{\scriptsize Gener de 2026, Escola d'Enginyeria (UAB)}
}

\renewcommand{\headrulewidth}{0pt}
\renewcommand{\footrulewidth}{0pt}
\pagestyle{fancy}

\twocolumn[\begin{@twocolumnfalse}

\maketitle

\thispagestyle{primerapagina}

\begin{center}
\parbox{0.915\textwidth}
{\sffamily
\textbf{Resum--}
Engraginy és un videjoc 3D que busca simular mecaniques de transmisió mecanica, fent ús d'engranatges, eixos, i altes mecanismes amb l'objectiu de construïr cadenes de producció i fabriques. El nucli del projecte és el sistema de potencia i transmisió mecànica, basat en grafs que fa ús de calcul vectorial i senyals per actualitzar el sistema quan és requereix. Ademés d'un sistema de transport de materials basat en simulacions físiques. Tots aquests sistemes disenyats i pensats per oferir una experiencia totalment reactiva.

\textbf{Paraules clau-- } Godot, Videojoc, Grafs, GDScript, Logística, Simulació 3D.

\bigskip

\textbf{Abstract--}
Engraginy is a 3D video game that seeks to simulate mechanical transmission mechanics, utilizing gears, shafts, and other mechanisms with the goal of building production chains and factories. The core of the project is the power and mechanical transmission system, based on graphs that use vector calculus and signals to update the system as required. Additionally, it features a material transport system based on physical simulations. All these systems are designed and intended to offer a fully reactive experience.

\textbf{Keywords-- } Godot, Videogame , Graphs, GDScript, Logistics, 3D Simulation.
}

\bigskip

{\vrule depth 0pt height 0.5pt width 4cm\hspace{7.5pt}%
\raisebox{-3.5pt}{\fontfamily{pzd}\fontencoding{U}\fontseries{m}\fontshape{n}\fontsize{11}{12}\selectfont\char70}%
\hspace{7.5pt}\vrule depth 0pt height 0.5pt width 4cm\relax}

\end{center}

%\bigskip
\end{@twocolumnfalse}]

\blfootnote{$\bullet$ E-mail de contacte: biel.alavedra@gmail.com}
\blfootnote{$\bullet$ Menció: Computació}
\blfootnote{$\bullet$ Treball tutoritzat per: Enric Marti Godia}
\blfootnote{$\bullet$ Curs 2025/26}

\section{Introducció - Context del treball}

\lettrine[lines=3]{E}{l} projecte Engraginy neix de la voluntat d'explorar mecàniques de joc més físiques dins del gènere de l'automatització. Mentre que títols referents com \textit{Factorio} i \textit{Satisfactory} utilitzen l'electricitat com un recurs binari o simplificat, aquest treball proposa una xarxa de potència mecànica on cada component (engranatges, eixos, cintes) afecta el rendiment de les cadenes de producció segons les lleis de la cinemàtica bàsica.

L'objectiu principal és la creació d'un sistema modular i extensible en el motor Godot 4 que permeti al jugador construir xarxes de potència complexes. El repte tecnològic resideix en la propagació dinàmica de la rotació en un entorn 3D, requerint l'ús de càlculs vectorials per determinar la viabilitat de les connexions, i les direccions de transmisió.

\section {Motivació}
La meva motivació per desenvolupar aquest projecte és principalment la d’aprendre com funciona el desenvolupament d’un videojoc. Els meus referents a l’hora de desenvolupar el projecte són videojocs com ‘Factorio’[5], ‘Satisfactory’[6], ‘Dyson sphere program’[7], ‘Shapez’[8], i ‘Minecraft: Create Mod’[9] (una modificació del joc no oficial).
D’aquests referents amb els quals tinc més experiència són DSP i Satisfactory, dos videojocs que m’han enganxat molt i m’han mostrat què entretingut i mentalment activador pot arribar a ser el gènere. El DSP és un joc molt més ambiciós, es comença en un petit satèl·lit dins d’un cúmul d’estrelles generat aleatòriament, i acaba quan el jugador construeix una esfera de Dyson i viatja entre sistemes solars. Tot això amb una perspectiva isomètrica que dona molt control i molta visió de tot allò que construeixes. En el Satisfactory l’objectiu és completar un ascensor espacial. Es comença en un planeta (el qual sempre és el mateix, el mapa és prefixat). Aquest joc és en primera persona, i totes les màquines que es construeixen són gegantines. Això impedeix una visió completa d’allò que construeixes. En quantitat de màquines les construccions són sempre més petites i molt més lentes de construir.

Dels dos jocs vull fer una cosa més similar a Satisfactory, ja que aquesta visió en primera persona dona un control més granular de les construccións a petit nivell, i encaixa molt millor amb les mecaniques que vull incorporar de Minecraft: Create mod, en la majoria de jocs de l’estil una part molt important és la generació d’energia, no serveix de res fer fàbriques més grans si no es tenen els recursos per mantenir-les, tant DSP, Factorio i Satisfactory utilitzen energia elèctrica, sigui solar, eòlica, carbó o nuclear. Tots aquests generadors s’acaben connectant a la xarxa elèctrica per donar energia a totes les màquines. En canvi, el Minecraft: Create mod requereix que tot sigui alimentat per energia cinètica, fent ús d’eixos, engranatges grans i petits, cadenes, corretges de transmissió... Com el meu objectiu és fer un joc amb una ambientació prerevolució industrial, i no del tipus fantàstic, crec que aquesta és la millor forma de fer-ho. També això afegeix totes les mecàniques d’haver de connectar les màquines, no només a una velocitat concreta, doncs algunes seran direccionals, com les cintes per transportar material.

D’entre tots els gèneres de videojocs perquè he escollit automatització? Considero que
aquest és un gènere que segueix molt la filosofia d'un programador, o en general la de tots els enginyers: dividir i vèncer. L’inici d’aquests jocs és senzill, hi ha unes poques màquines i cal fer processos simples. Per exemple, amb un extractor de recursos automàtic i cal processar els recursos en brut per tal d’obtenir el recurs processat (de mena de ferro lingots a de ferro). Però això ràpidament canvia, agafant d’exemple el joc DSP. Si vols fer una placa de circuits es necessita un forn de fosa que converteixi el ferro en brut a plaques de ferro, un altre forn de fosa que converteixi coure en brut a plaques de coure i un assemblador que agafi aquests dos materials i els converteixi en la placa de circuits. Això és només un dels primers passos, on cada vegada es disposa de més i més materials diferents, processos diferents i els objectes requerits són cada vegada més complexes. Per força no es poden afrontar tots aquests problemes simultàniament, cal dividir els problemes en mòduls que es puguin replicar cada vegada que calgui per a resoldre aquell problema.

Dins de la comunitat de fans d’aquest gènere de videojoc hi ha molts enginyers i gent que li agrada molt optimitzar processos i fer construccions el màxim d’eficients possibles. Per tot això no només considero el gènere com a entretingut, sinó que també en certa manera és educatiu, doncs força al jugador tant sí com no a organitzar-se, pensar formes eficients de construir coses, com connectar les entrades i sortides de totes les fabriques, calcular què tanta entrada es necessita per la sortida objectiva i construir d’acord amb això. Per completar un joc d’aquests cal ser organitzat, metòdic i pensar en solucions segons el problema que es tingui.

\section {Estat de l'art}
Aquest és un genere relatativament modern, comparat a altres generes com els "First person shooter", "Point and click", Plataformes, Acció, Aventures, "Role playing game", etc. Tot i així desde la creació del gènere a voltants del 2010 han sortit grans videojocs que han infuluenciat en gran mesura la direcció del gènere.

\begin{itemize}
\item Factorio
\end{itemize}
Factorio va ser desenvolupat per l’estudi Txec Wube Software, el desenvolupament va
començar l’any 2012, l’any 2014 va entrar en accés anticipat i l’any 2020 va sortir la versió completa. L’any 2024 va sortir una expansió, «Factorio Space Age», que duplica el joc en contingut, ampliant el joc més enllà del final base. És un joc 2D amb vista top-down.
En aquest joc comences com un enginyer que s’ha estavellat contra un planeta alienígena, i l’objectiu del joc base és construir un coet i escapar del planeta. Per aconseguir-ho és necessari construir sobre el planeta un gran conjunt de fàbriques interconnectades per tal d’aconseguir tots els materials necessaris. El joc té un sistema de progressió tecnològica basada en unes «pocions de ciència», les quals són necessàries per investigar i desbloquejar nous materials i edificis per processar-los.
El joc també compte amb un sistema de combat basat en la contaminació, com més
maquines tens i més contamines el planeta més apareixen uns insectes enormes que venen
a destruir les teves construccions, per tant, necessites posar murs i torres defensives per evitar-ho. Aquest sistema és completament opcional i es pot desactivar per si un vol jugar més tranquil.

\begin{itemize}
\item Satisfactory
\end{itemize}
Satisfactory va ser desenvolupat per l’estudi Suec Coffe Stain Studios, va ser anunciat l’any 2018, llançat en accés anticipat l’any 2019 i finalment completat l’any 2024. És un joc 3D en primera persona.
En el joc ets un treballador de l’empresa FICSIT, i t’encarreguen construir un ascensor
espacial per tal de salvar la humanitat. En aquest joc tots els edificis són enormes, quan jugues i construeixes qualsevol cosa com veus les coses en primera persona tot sembla grandiós, i això fa una sensació al jugar molt diferent que els altres jocs, on ho veus tot des de dalt i tens molt més control de tot allò que construeixes. Que funcioni en primera persona també permet construir amb molta més verticalitat, fer grans fabriques amb múltiples plantes i sales.
Per tal de progressar en el joc has de suplir els materials requerits per l’ascensor espacial, això et desbloqueja la possibilitat de comprar noves tecnologies, per comprar-les has d'utilitzar materials comuns, els quals són necessaris per construir les parts per l’assessor espacial, per tant, no et veus forçat a construir coses que només serveixen per a un propòsit.

\begin{itemize}
\item Dyson sphere program
\end{itemize}
Dyson sphere program és desenvolupat per l’estudi xinès Youthcat Studio, el joc va ser llançat en accés anticipat l’any 2021, el joc encara no té anunciat una versió completa i segueix en desenvolupament. És un joc 3D en vista isomètrica.
El joc comença en quant arribes a un cúmul d’estrelles i aterres en un satèl·lit d’un gegant gasos en un dels sistemes solars, ets un robot encarregat de construir un gran conjunt de fàbriques per tal d’acabar construint una esfera de Dyson. L’estil de joc és molt similar a Factorio, tot i que el joc és 3D les construccions no tenen tanta verticalitat com si la té Satisfactory, pots construir cintes transportadores cap amunt i hi ha certs edificis que en pots posar múltiples un sobre l’altre, com baüls per emmagatzemar materials, però per la resta és majoritàriament sobre pla el joc. Ràpidament, s’acabarà l’espai i els recursos i hauràs d’expandir-te cap a altres planetes i sistemes solars, en cerca de més recursos bàsics i recursos avançats.
El sistema de progressió és similar a Factorio, en compres de «pocions de ciència», hi ha uns «cubs de ciència» que hauràs de crear i després utilitzar per a la investigació de noves tecnologies.
El joc també compte amb un sistema de combat basat en uns enemics que t’ataquen per onades i cada vegada són més agressius depenent de la quantitat d’energia que generes. Molt similar a Factorio.

\begin{itemize}
\item Shapez
\end{itemize}
Shapez és un videojoc de codi obert desenvolupat per Tobias Springer, va ser llençat l’any 2020. És un joc 2D amb una vista top-down i uns gràfics molt minimalistes. En el joc no controles a cap personatge, ni tens cap història ni repte a completar, és un joc basat per nivells on un Hub central demana una sèrie de figures geomètriques concretes.
Completar aquests requisits et permet completar els nivells i aquests nivells  desbloquegen noves formes de tractar aquestes figures, ja sigui tallar-les, rotar-les, pintar-les, ajuntar-les o estampar-les una sobre l’altre. Comences amb cercles, quadrats i estrelles i d’això surten milers de combinacions disponibles. El repte en aquest joc és organitzar tota la logística de les fàbriques i ser capaç de fer dissenys escalables, per reduir els temps d’espera per completar el nivell, perquè si demanen 1000 d’una peça i només en fas 1 al segon hauràs d’esperar més de 15 min.
Tots els recursos són il·limitats i construir i destruir edificis no té cost, és una experiència molt més relaxant comparat a la resta de jocs, això no significa, ni molt menys, que completar el joc sigui senzill.

\begin{itemize}
\item Minecraft: Create Mod
\end{itemize}
El mod Create de Minecraft és desenvolupat per «Simibubi», la primera versió del mod va
sortir l’any 2019, i a l’any actual 2025 ja van per la versió 6.0. Minecraft és un joc 3D en primera persona, on estàs en un món pràcticament infinit de cubs, on tens llibertat per construir o trencar tot el que vulguis.
Aquest mod funciona per sobre de Minecraft vanilla, per tant, tot el que és possible en el joc base aquí funciona igual, no modifica les bases del joc original, pots jugar al Minecraft com si res del mod existís.

Igual que Minecraft vanilla el mod no et demana res específic, ni complir amb cap objectiu ni completar cap història, només és una gran col·lecció d’eines disponibles per fer el que vulguis.
Hi ha una sèrie de blocs que actuen com a recol·lectors de recursos per tal d’automatitzar tots els processos que vulguis. Aquí aconseguir recursos és molt diferent, perquè no tots els recursos del joc es poden aconseguir de forma il·limitada, pots aconseguir moltes coses, però hi haurà límits i coses que s’hauran de fer de forma específica, un exemple: Si el nostre objectiu és aconseguir lingots de ferro haurem de passar per múltiples processos, utilitzant aigua i lava un al costat de l’altre podem fer un generador de roca, utilitzant un trepant podem anar eliminant aquesta roca i fer que caigui en una pedra de molí, aquesta trencarà la roca per convertir-la en grava, aquesta la passarem per un procés de rentat amb aigua i això amb una baixa possibilitat ens pot donar sílex i/o pepes de ferro, si separem aquests dos materials llavors podem compactar 9 pepes per fer un lingot. Es pot veure que són processos molt més llargs i més complexos que els altres jocs. Fent cadenes similars es poden aconseguir pràcticament tots els materials del joc.
Aquest mod té un dels sistemes d’energia més interessants que he provat mai, no existeix l'opció de fer energia elèctrica, tot és energia cinètica i força de rotació, fent servir rodes d’aigua, molins de vent o de forma més avançada motors de vapor, pots generar energia rotacional i a través d'eixos, corretges, engranatges, i caixes de canvis has de modular tant la velocitat com la direcció de gir que requereixen les màquines perquè funcionen com un necessita. Aquest mateix sistema és el que vull portar al meu joc.

De tots els jocs esmentats n’hi ha dos que destaquen per sobre la resta, aquests són Factorio i Satisfactory. Dins del gènere aquests dos són els més representatius i els més recomanats. Factorio és més complex tant visualment com pel que fa als sistemes i sempre es recomana més si tens o experiència amb el gènere o amb altres jocs de gestió de recursos, Satisfactory és més senzill d’entendre per què ho veus tot en primera persona i tota la GUI és molt més accessible. Són dos jocs que per a completar-los són mínim 100 hores, si jugues per primera vegada, i això només per completar l’objectiu principal, molts jugadors es posen objectius ells mateixos i fan construccions i sistemes absurdament enormes.


\section{Arquitectura del Sistema}

El programari s'ha estructurat seguint els principis de la programació orientada a objectes (POO) mitjançant el sistema de nodes de Godot. La classe base \texttt{Building.gd} defineix els comportaments d'interacció, mentre que \texttt{PowerNode.gd} hereta aquesta base per incloure la lògica de ports i connexions. Engranatges, eixos, generadors i maquines hereten d'aquest \texttt{PowerNode.gd} per tal de definir més concretament les propietats especifiques de cada cas.
Ja que s'ha fet ús de Godot el projecte havia d'estar estructurat internament tal i com ho requereix el motor, és imperatiu fer us del sistema de nodes i escenes per crear tots els objectes i parts del videojoc. Aixó ens deixa mab dues grans parts, l'estructura del programari, basada en POO i l'estructura interna de tots els objectes i escenes que són un conjunt de nodes predeterminats i nodes amb una implementació propia.

\subsection{Nodes i escenes}
Abans de parlar de la meva implementació tècnica és important explicar com funciona el sistema de nodes i escenes de Godot.
Cada element del joc és un node amb unes propietats i funcions diferents. Hi ha tres famílies de nodes principals, nodes 3D, nodes 2D i nodes de GUI. Dins de la família de nodes 3D tenim exemples com:
\begin{itemize}
\item Camera3D
\item StaticBody3D
\item MeshInstance3D
\item CollisionShape3D
\item RayCast3D
\item etc.
\end{itemize}
Cada un d’aquests nodes és una classe diferent amb les seves propietats, però totes hereten de la classe Node3D. Els nodes de GUI ens permeten crear interfícies d’usuari on utilitzarem l’estructura jeràrquica per encapsular cada element d'interfície com ara:
\begin{itemize}
\item Container
\item Label
\item TextureRect
\item Button
\item Panel
\item ItemList
\item etc.
\end{itemize}
Les escenes seran definides per l’usuari i són una col·lecció de nodes de qualsevol tipus, una escena pot ser el jugador, la interfície d’usuari o un nivell sencer d’un joc. Depèn de cada desenvolupador decidir com organitzar. Aquestes escenes es poden afegir a altres escenes per poder així evitar treballar amb grans problemes a la vegada, i tenir-ho tot en mòduls més petits.

\begin{figure}[!h]
\centering
    \includegraphics[width=0.45\textwidth]{img/PlayerScene.png}
    \caption{Escena del jugador}
    \label{fig:player-scene}
\end{figure}

En l'escena que conté el jugador, Fig. \ref{fig:player-scene}, tenim com a node pare un "PlayerCharacter" (node propi, hereta de la clase base "CharacterBody3D"), amb 5 fills, el punt d'anclatge de la camera ("CameraHolder"), la caixa de col·lisions del personatge ("Hitbox"), els "RayCast3D" ("Raycasts"), el "Model" i la màquina d’estats encarregada del control del personatge. Aquesta màquina d’estats està iniciada amb nodes buits, aquests nodes són la classe base per a tots els objectes de Godot, i no tenen propietats especials. Podem veure que en el lateral tenim una icona amb un pergamí, això ens indica que hi ha un script vinculat al node, si està en gris vol dir que aquest script és heretat, compartit amb altres nodes de la mateixa classe. Si és blanc ens indica que el node té un script únic.

\section{Lògica de Transmissió Mecànica}
El sistema de transmisió mecànica és una adaptació del sistema existent a Minecraft: Create. Per tant, he volgut simular el seu funcionament, a continuació faré un llistat de les especificacions del sistema, i com s’hauria de comportar sota cada cas específic.
\begin{itemize}
\item Tota connexió d’un eix ha de mantenir direcció i velocitat.
\item Tota connexió d’un engranatge ha d’invertir la direcció.
\item Connectar un engranatge petit a un gran a través de les dents de l’engranatge ha de duplicar velocitat, a l’invers la velocitat es divideix.
\item La xarxa no pot superar el límit d’energia subministrada per tot el conjunt de generadors connectats.
\item En cas de superar el límit tot el sistema s’ha d’aturar a l’instant.
\item En cas de tornar a estar per sota del límit, el sistema ha d’entrar en funcionament a l’instant.
\item En cas que un node tingui incoherències en els seus ports connectats s’ha d’eliminar automàticament.
\end{itemize}
<FIG ./img/ConnectionsExample.png>
Figura. Mostra de possibles connexions.

La classe principal d’aquest sistema és "PowerNodes". D’aquesta classe hereten tots els elements de la xarxa que afegirem. Cada "PowerNode" tindrà mínim un "PowerNodePort", aquests ports són cubs ubicats allà on es vol que el node tingui un punt de connexió. Aquest port té quatre propietats molt importants: la direcció de gir respecte al node, el modificador de velocitat, el tipus de port que és, i a quins altres ports es pot connectar.


\begin{figure}[!h]
\centering
    \includegraphics[width=0.45\textwidth]{img/PowerNodePort.png}
    \caption{Exemple de port d'un eix}
    \label{fig:shaft_port}
\end{figure}
En la figura \ref{fig:shaft_port}, podem veure com estan configurats els ports d’un eix bàsic. Podem veure respectivament les propietats: modificador de velocitat ("Ratio Multiplier"), direcció de gir ("Direction Flipper"), tipus de port ("Type"), i connexions permeses ("Allow Ports"). Els tipus de port implementats són: "SHAFT\_END", "COG\_SMALL", "COG\_BIG" i "BELT". En cas d’implementar un element que ens permeti invertir gir hauríem de tenir dos ports, un com el de la figura \ref{fig:shaft_port}, i l’altre amb el "Direction Flipper" a -1, si volguéssim implementar un element que ens permetis duplicar la velocitat de gir hauríem de modificar el "Ratio Multiplier" a 2.

\begin{figure}[!h]
\centering
    \includegraphics[width=0.45\textwidth]{img/ShaftScene.png}
    \caption{Exemple de l'escena d'un eix}
    \label{fig:shaft_scene}
\end{figure}
En la figura \ref{fig:shaft_scene} veiem l’exemple de l’eix, amb els dos ports col·locats un a cada extrem de l’objecte.

Aquests ports hereten de la classe base de Godot "Area3D". Això és perquè aquesta classe base té ja té implementat un mètode que en cas d’un altre "Area3D" entri en contacte podem connectar una funció que com argument ens doni l’àrea que ha entrat.

Quan detectem aquesta connexió comprovem que l'altra àrea sigui un "PowerNodePort" i que la connexió sigui vàlida. En cas de connexió vàlida llancem el senyal "network\_changed", que indica al "PowerGridManager" que hi ha hagut un canvi en la xarxa a la qual pertany el node, i per tant s’ha de recalcular la xarxa.

Aquest procés de càlcul comença amb un algoritme "Breadth-first search" (BFS), es podria utilitzar també un "Depth-first search" (DFS) en aquest cas, per tal de trobar tots els nodes connectats. Quan volem trobar tots els nodes connectats no hi ha diferència entre utilitzar un BFS o un DFS, però més endavant s'ha d'utilitzar forçosament un BFS, per tant, ja tenim la funció implementada. Una vegada tenim tots aquests nodes busquem quins són els generadors que hi ha a la xarxa, i és a partir d’aquests generadors que comencem a propagar tots els canvis. Per propagar els canvis apliquem un altre BFS. En aquest cas ha de ser aquest algoritme, ja que interessa que els canvis es propaguin capa a capa, que es calculin abans els nodes més pròxims per així detectar els possibles conflictes en els extrems. Per cada node definim la seva velocitat com la calculada en el moment d’afegir-lo a la llista de "per visitar", i iterem per les seves connexions per tal de calcular la velocitat que haurien de tenir per no generar conflictes. En cas de no tenir la connexió una velocitat ja assignada li assignem la calculada. Si ja tenia una velocitat assignada ens assegurem que no hi hagi conflictes. En cas de conflicte eliminem la peça i tornem a començar el procés de càlcul de la xarxa. Si no hi ha conflicte afegim la connexió al llistat de "per visitar". Finalment, quan hem acabat amb el càlcul de les velocitats recorrem tota la xarxa per comprovar que la potència és prou per mantenir la xarxa activa. Tots els generadors sumen les seves potències i la resta d’elements resten.

La part més complexa d’aquest procés és el càlcul de velocitats. Els nodes poden estar situats en qualsevol punt del món i en qualsevol direcció. Per exemple un eix sense girar connectat a un de girat 180 graus semblen el mateix, però si es fa una assignació directa de velocitat, el segon eix girarà en sentit contrari. Per solucionar això fem una sèrie de càlculs amb els vectors de direcció de cada element per comprovar com s’ha de comportar aquella connexió. Per fer aquest càlcul seguirem una sèrie de passos, aquests pasos no s'executen sempre tots, depenent dels resultats d'operacions anteriors poden cambiar els pasos que s'executen.

\begin{enumerate}
\item $W_{input} = W_{conn} \times R_{conn} \times D_{conn}$
\item $Alg = sgn(\vec{A} \cdot \vec{B})$
\item $Alg = sgn(\vec{S}_{dir} \cdot (1, 1, 1))$
\item $\vec{V} = (\vec{P}_{conn} + \vec{C}_{conn}) - (\vec{P}_{local} + \vec{C}_{local})$
\item $Alg = sgn((\vec{A} \times \vec{V}) \cdot (-\vec{B} \times \vec{V}))$
\item $W_{res} = \frac{W_{input} \times D_{local} \times Alg}{R_{local}}$
\end{enumerate}
En la primera fórmula obtenim la velocitat d’entrada aparent, sense tenir en compte l’alineació dels dos nodes. Els paràmetres són els següents: Wconn és la velocitat de rotació del node connectat, «Rconn» és el multiplicador de velocitat del port del node connectat i «Dconn» és la direcció del port del node connectat.

En la segona calculem l’alineació de l’eix de rotació propi, A, i l’eix de rotació del port connectat, B. L'alineació es calcula utilitzant el producte escalar entre els dos eixos de rotació. A partir d’aquest punt poden passar dues coses: que el valor Alg sigui 0 o 1/-1.

En cas que Alg sigui 1/-1 significarà que els dos ports són paral·lels. Si els dos ports són del tipus "SHAFT\_END" la velocitat del port serà la Winput multiplicada pel signe de Alg. En cas que els ports siguin del tipus "COG" la velocitat serà Winput multiplicada pel negatiu de signe de Alg, per tal d’invertir la direcció de gir.

En el cas que Alg sigui 0 voldrà dir que els dos ports són perpendiculars, aquest es dona quan connectem dos engranatges grans en perpendicular, o quan un eix està connectat a una cinta mecànica. En el segon cas per obtenir la direcció correcta apliquem la tercera fórmula, on Sdir és la direcció de gir de l'eix. En el primer cas, dos engranatges grans en perpendicular, apliquem les fórmules 4 i 5. Amb la fórmula 4 obtenim un vector que va des del centre del node connectat al centre del node actual, Pconn/Plocal són les posicions de l'element en l'escena i Cconn/Clocal és un desplaçament per als casos on el centre de gir del node no correspon en la posició en escena. Finalment calculem l'alineació.

Per acabar, aquesta velocitat Winput, que representa la velocitat del port la convertim a la velocitat interna que ha de tenir el node, utilitzant la sisena fórmula desfem el càlcul fet en la primera, però utilitzant els valors del nostre port i multipliquem per l'alineació final, Alg. Aquest és el valor que retornem perquè el "PowerGridManager" comprovi si hi ha conflicte.

\section{Logística i Transport d'Ítems}

\section{Interficies d'usuari}

\section{Construcció i interacció amb objectes}

\section{Guardar i carregar escenaris}

\section{Conclusions}

\section*{Agraïments}

\begin{thebibliography}{11}
\bibitem{godot} Godot

\end{thebibliography}

\appendix

\section*{Apèndix}


\end{document}
